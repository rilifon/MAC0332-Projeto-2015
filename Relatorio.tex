\documentclass[a4paper]{article}
\usepackage[english]{babel}
\usepackage[utf8]{inputenc}

\usepackage{titling}
\newcommand{\subtitle}[1]{%
  \posttitle{%
    \par\end{center}
    \begin{center}\large#1\end{center}
    \vskip0.5em}%
}

\newcommand{\lra}{\leftrightarrow}
\newcommand{\ra}{\rightarrow}

\title{BudgetChef}
\subtitle{Fase 0 - MAC0332}

\author{
	Guilherme Schützer - 8658544\\
	Renato Geh - 8536030\\
	Ricardo Lira - 8536131\\
	Tomás Paim - 7157602\\
}

\date{15/09/2015}

\begin{document}
\maketitle

\section{Proposta de projeto}

Criação de um aplicativo mobile para gerenciar os produtos gastronômicos de sua casa, determinar quantidades e avisar sobre datas de validades próximas do vencimento, recomendar receitas que utilizam produtos disponíveis ou determinar o que falta para abastecer seu lar.

\section{Viabilidade do sistema}

O aplicativo tem um grande escopo de público alvo, cobrindo desde jovens adultos a responsáveis por famílias. A flexibilidade do aplicativo permite que qualquer pessoa que tenha interesse em descobrir novas receitas, seja por necessidade ou por curiosidade, possa ter com facilidade uma receita disponível para se fazer na hora.

\par Por ser um aplicativo intuitivo e sem custo para o usuário padrão, BudgetChef é acessível a qualquer um que tenha um smartphone compatível com o requerido pelo aplicativo.

\par A interface do aplicativo permite navegar com facilidade por todas as suas funcionalidades. Envio de novas receitas é feita de forma intuitiva e simples, apenas sendo necessário ter uma conta válida, além de o próprio aplicativo ajudar no que se deve adicionar a sua nova receita, sugerindo novas informações conforme o usuário as vai colocando. O aplicativo também sugere receitas que sejam similares àquelas que ele havia pesquisado ou enviado.

\end{document}
