\documentclass[a4paper]{article}
\usepackage[english]{babel}
\usepackage[utf8]{inputenc}
\usepackage{hyperref}
\usepackage{booktabs}

\usepackage{titling}
\newcommand{\subtitle}[1]{%
  \posttitle{%
    \par\end{center}
    \begin{center}\large#1\end{center}
    \vskip0.5em}%
}

\title{\textbf{BudgetChef - Dicionário de Dados}}
\subtitle{Fase 1 - MAC0332}

\author{
	Guilherme Schützer - 8658544\\
	Renato Geh - 8536030\\
	Ricardo Lira - 8536131\\
	Tomás Paim - 7157602\\
}

\date{08/10/2015}

\begin{document}
\maketitle

\subsubsection{} A primeira relação é de receitas. Contém os campos não-multivalorados necessários para descrever uma receita e avaliá-la.


\begin{center}
\begin{tabular}{ l l l l l }
  \multicolumn{5}{l}{\textbf{RECEITAS}} \\
  \hline
  \textbf{Chave} & \textbf{Campo} & \textbf{Descrição} & \textbf{Tipo} & \textbf{Tamanho} \\
  \cline{1-5}
  Primária & REC\_CODIGO & Código da Receita & Inteiro & 8 \emph{bits}  \\
   & REC\_NOME & Nome da Receita & Texto & 255 \\
   & USU\_LOGIN & Usuário autor da Receita & Texto & 255 \\
   & REC\_DATA & Data de Submissão & Data & 6 \\
   & REC\_TEMPO & Tempo de preparo & Tempo(Inteiro) & 4 \\
   & REC\_PORC & Quantas porções serve & Inteiro & 8 \emph{bits} \\
   & REC\_AVAL & Quantidade de avaliações & Inteiro & 8 \emph{bits} \\
   & REC\_MEDIA & Média das avaliações & Decimal & 5,00 \\
   & REC\_VISIT & Quantidade de visitas & Inteiro & 8 \emph{bits}
\end{tabular}
\end{center}

\vfill

\subsubsection{} A relação de Ingredientes se refere aos ingredientes necessários para cada receita. Como cada receita pode ter mais de um ingrediente e cada ingrediente pode estar em mais de uma receita, os dois campos constituem a chave primária.

\begin{center}
\begin{tabular}{ l  l  l  l l }
  \multicolumn{5}{l}{\textbf{INGREDIENTES}} \\
  \hline
  \textbf{Chave} & \textbf{Campo} & \textbf{Descrição} & \textbf{Tipo} & \textbf{Tamanho} \\
  \cline{1-5}
  Primária & REC\_CODIGO & Código da Receita & Inteiro & 8 \emph{bits}  \\
  Primária & ING\_NOME & Nome do Ingrediente & Texto & 255 \\
   & ING\_QUANT & Quantidade necessária & Decimal & 16 \emph{bits} \\
   & ING\_UNI & Unidade (colheres, gramas...) & Texto & 255
\end{tabular}
\end{center}

\newpage

\subsubsection{} Relação de Usuário. Contém as informações básicas e para estatísticas de receitas.

\begin{center}
\begin{tabular}{ l  l  l  l l }
  \multicolumn{5}{l}{\textbf{USUÁRIOS}} \\
  \hline
  \textbf{Chave} & \textbf{Campo} & \textbf{Descrição} & \textbf{Tipo} & \textbf{Tamanho} \\
  \cline{1-5}
   Primária & USU\_LOGIN & Nome de \emph{login} & Texto & 255 \\
   Candidata & USU\_EMAIL & Endereço de \emph{e-mail} do usuário & Texto & 255 \\
   & USU\_NOME & Nome completo do Usuário & Texto & 255 \\
   & USU\_MEDIA & Média de todas as avaliações & Decimal & 5,00 \\
   & USU\_RECS & Número de receitas submetidas & Inteiro & 8 \emph{bits}
\end{tabular}
\end{center}

\vfill

\subsubsection{} Relação de Estoques. Parecida com a relação Ingredientes, mas referente ao estoque de cada usuário e não às receitas. Contém também informações de validade.

\begin{center}
\begin{tabular}{ l  l  l  l l }
  \multicolumn{5}{l}{\textbf{ESTOQUE}} \\
  \hline
  \textbf{Chave} & \textbf{Campo} & \textbf{Descrição} & \textbf{Tipo} & \textbf{Tamanho} \\
  \cline{1-5}
  Primária & USU\_LOGIN & \emph{Login} do Usuário & Texto & 255 \\
  Primária & ING\_NOME & Nome do Ingrediente em estoque & Texto & 255 \\
   & EST\_QUANT & Quantidade em estoque & Decimal & 16 \emph{bits} \\
   & EST\_UNI & Unidade (colheres, gramas...) & Texto & 255 \\
   & EST\_DATA & Data de estocagem & Data & 6 \\
   & EST\_VAL & Data de validade & Data & 6
\end{tabular}
\end{center}

\vfill

\subsubsection{} Relação das Famílias dos Ingredientes. Útil para praticidade de entrada.

\begin{center}
\begin{tabular}{ l  l  l  l l }
  \multicolumn{5}{l}{\textbf{FAMÍLIAS}} \\
  \hline
  \textbf{Chave} & \textbf{Campo} & \textbf{Descrição} & \textbf{Tipo} & \textbf{Tamanho} \\
  \cline{1-5}
  Primária & FAM\_NOME & Nome da Família (carne, carboidrato...) & 255 \\
  Primária & ING\_NOME & Nome do Ingrediente & Texto & 255 \\
  & FAM\_DESC & Breve descrição da Família & Texto & 255
\end{tabular}
\end{center}

\newpage

\subsubsection{} Relação das Categorias das Receitas. Categorias podem ser usadas como filtro na busca de receitas.

\begin{center}
\begin{tabular}{ l  l  l  l l }
  \multicolumn{5}{l}{\textbf{CATEGORIA}} \\
  \hline
  \textbf{Chave} & \textbf{Campo} & \textbf{Descrição} & \textbf{Tipo} & \textbf{Tamanho} \\
  \cline{1-5}
  Primária & ATR\_NOME & Nome do Atributo & Texto & 255 \\
  Primária & REC\_CODIGO & Código da receita com o atributo & Inteiro & 8 \emph{bits} \\
   & ATR\_DESC & Breve descrição do Atributo & Texto & 255
\end{tabular}
\end{center}

\subsubsection{} Relação de amizades; atributo multivalorado.

\begin{center}
\begin{tabular}{ l  l  l  l l }
  \multicolumn{5}{l}{\textbf{AMIZADE}} \\
  \hline
  \textbf{Chave} & \textbf{Campo} & \textbf{Descrição} & \textbf{Tipo} & \textbf{Tamanho} \\
  \cline{1-5}
  Primária & USU\_LOGIN & Login de um usuário & Texto & 255 \\
  Primária & USU\_AMIG & Login de seu amigo & Texto & 255
\end{tabular}
\end{center}

\end{document}
